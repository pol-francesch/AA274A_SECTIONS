\documentclass{article}

\usepackage{amsmath}
\usepackage{amsthm}
\usepackage{amssymb}
\usepackage{bbm}
\usepackage{fancyhdr}
\usepackage{listings}
\usepackage{cite}
\usepackage{graphicx}
\usepackage{enumitem}
\usepackage[margin=1cm]{caption}
\usepackage{subcaption}
\usepackage{tcolorbox}
\usepackage{color}
\definecolor{editorGray}{rgb}{0.95, 0.95, 0.95}


\lstset{%
    % Basic design
    backgroundcolor=\color{editorGray},
    basicstyle={\small\ttfamily},   
    frame=l,
    % Line numbers
    xleftmargin={0.75cm},
    numbers=left,
    stepnumber=1,
    firstnumber=1,
    numberfirstline=false,
    }
\lstset{
    literate={~} {$\sim$}{1}
}

\newenvironment{claim}[1]{\par\noindent\underline{Claim:}\space#1}{}
\newenvironment{claimproof}[1]{\par\noindent\underline{Proof:}\space#1}{\hfill $\blacksquare$}

\oddsidemargin 0in \evensidemargin 0in
\topmargin -0.5in \headheight 0.25in \headsep 0.25in
\textwidth 6.5in \textheight 9in
% set the header up
\fancyhead{}
\fancyhead[L]{Stanford University}
\fancyhead[R]{Principles of Robot Autonomy I - Fall 2021}

%%%%%%%%%%%%%%%%%%%%%%%%%%
\renewcommand\headrulewidth{0.4pt}
\setlength\headheight{15pt}
\input{preamble}

\usepackage{outlines}

\usepackage{gensymb}
\usepackage{algorithm}
\usepackage{algpseudocode}


\newcommand{\ssmargin}[2]{{\color{blue}#1}{\marginpar{\color{blue}\raggedright\scriptsize [SS] #2 \par}}}

\newcommand{\x}{\mathbf{x}}

\setlength{\parindent}{0in}

\title{AA 274A: Principles of Robotic Autonomy I \\Section 5 (in-person): Implementing Point-to-Point Navigation}
\date{}

\setlength{\textfloatsep}{10pt} % Change vertical space after algorithm block


\begin{document}

\maketitle
\pagestyle{fancy}
\vspace{-1.25cm}
Our goals for this section: 
\begin{enumerate}
    \item Learn how to read and understand source code for more complex ROS nodes
    \item Test controllers from homework on a real robot
    \item Learn how to design custom launch files
\end{enumerate}

\section{Point to point motion around obstacles}
% As you saw in the last section, one of the nodes that the \texttt{turtlebot3\_bringup\_jetson\_pi} launch file starts is \texttt{gmapping}, which uses the LIDAR readings to perform SLAM (simultaneous localization and mapping), giving us an \textit{occupancy grid} map of the environment around the robot, as well as an estimate of the robots position within this map.\footnote{We'll be covering SLAM in class next week and learning how gmapping is able to do this, but for now it's fine to think of it as magic.}
Before starting this section, please pull and clean the \texttt{~/catkin\_ws/src/asl\_turtlebot} repository in BOTH computer and robot:
\begin{lstlisting}
git pull
git reset --hard
git clean -fdx
\end{lstlisting}

As you saw in the last section, one of the nodes that the section launch file started was \texttt{gmapping}, which uses LIDAR readings to perform simultaneous localization and mapping (SLAM), giving us an \textit{occupancy grid} map of the environment around the robot, as well as an estimate of the robots position within this map.\footnote{We'll be covering SLAM in class next week and learning how gmapping is able to do this, but for now it's fine to think of it as magic.}

A key ability of autonomous agents is the ability to navigate from point to point in the presence of obstacles. Today we'll be implementing a navigator ROS node and testing this functionality on the turtlebots!

% First, let's git pull the latest changes to the \texttt{asl\_turtlebot} package which contains the starter code for the navigator node in \texttt{scripts/navigator.py}. As before, from the \texttt{asl\_turtlebot} directory, run

% \begin{minted}{bash}
% git pull
% \end{minted}

% If you're using docker, you will need to also update your docker config. From the \texttt{aa274-docker} directory, run

% \begin{minted}{bash}
% git pull
% ./build_docker.sh
% \end{minted}


%Next
Open \texttt{scripts/navigator.py} within the \texttt{asl\_turtlebot} catkin package in a text editor, read the provided code, and think about how the node works.

\textbf{Problem 1: What topics does the navigator subscribe to? What is the purpose of each of these topics? What topics does it publish to, and why?}

The navigator uses a state machine to switch between different modes of operation. Carefully read the functions \texttt{run} and \texttt{publish\_control}.

\textbf{Problem 2: Describe what each mode of the state machine does, and intuitively when the node switches between modes.}

You may have noticed that the code logic is similar to the strategy we used in HW2: planning around obstacles using A*, and using a combination of the pose controller and tracking controller to track the planned path. In fact, this code calls functions that you wrote in your previous homeworks. 
\\
\textbf{NOTE}: The navigator node can be run from the robot \textit{or} from the aa274 laptop(preferred). Choose which device will be running the navigator, and copy files over accordingly.

Copy over the following files to \texttt{asl\_turtlebot/scripts/controllers/}

\begin{lstlisting}
P2_pose_stabilization.py
P3_trajectory_tracking.py
\end{lstlisting}

and the following from HW2 to \texttt{asl\_turtlebot/scripts/planners/}

\begin{lstlisting}
P1_astar.py
\end{lstlisting}

You will also need to make some small modification to {\tt P1\_astar.py}. Find and replace the initialization of two variables as following, and test it locally(on your own computer, with \texttt{resolution = 0.1}) to make sure it is working as expected.

\begin{lstlisting}
self.est_cost_through = {}
self.cost_to_arrive = {}
\end{lstlisting}

Also, edit \texttt{scripts/planners/path\_smoother.py} and copy over the function \texttt{compute\_smoothed\_traj} from HW2's \texttt{P3\_traj\_planning.py}.

%\textbf{For SCPD students:} 
If you need to, use the \texttt{scp} command to copy these files to the robot. An alternative is to use VScode Remote extension to log into robot and modify it directly on robot.


\begin{lstlisting}
# Below, you need replace "henry" with the name of your robot 
# and fill in the '...' with the correct file path!
scp <file> aa274@henry.local:~/catkin_ws/src/asl_turtlebot/scripts/...
\end{lstlisting}

Now, we're ready to test the framework on the robot!

% \subsection{On Campus Students}
% Power on the robot, and have one person in your group ssh into it.
% Ensure it's running the latest version of our codebase by running the following commands
% \begin{minted}{bash}
% roscd asl_turtlebot
% git pull
% \end{minted}

% Finally, run \texttt{turtlebot3\_bringup\_jetson\_pi.launch} as we did in last section.

% From your computer, (after running \texttt{rostb3}), \textbf{one person per robot} should run the following to start the navigator
% \begin{minted}{bash}
% rosrun asl_turtlebot navigator.py
% \end{minted}
% Again, only one navigator should be running per robot, so take turns to test your code!

% \subsection{SCPD Students}
On the robot (i.e. after ssh-ing in the robot), run

\begin{lstlisting}
roslaunch asl_turtlebot asl_turtlebot_core.launch
\end{lstlisting}

Then, in a new terminal (after running \texttt{rostb3} or ssh-ing into the robot, depending on where you've copied your navigator code), run

\begin{lstlisting}
rosrun asl_turtlebot navigator.py
\end{lstlisting}

\section{Running the Navigator}

From a new terminal, open \texttt{rviz}. 
% As before, if using docker, run the following command from the \texttt{aa274-docker} directory.
% \begin{minted}{bash}
% ./run.sh --display 1 --rosmaster <lowercase_robot_name>.local rviz
% \end{minted}

Add relevant topics to the display - the main ones we'll need are \texttt{/map}, the TF transform tree, and the path topics \texttt{/planned\_path} and \texttt{/cmd\_smoothed\_path}. 
%The \texttt{/camera} topic will also allow you you see what the robot sees as it navigates through the maze.

Create a new catkin package (like how we made one in Section 2) with the name \texttt{section5} and save the rviz configuration as \texttt{my\_nav.rviz} in \texttt{section5/rviz/my\_nav.rviz}.

\textbf{Problem 3: What is the command to create a new package? (Hint: Take a look at Section 2's handout for a starting point). What do each of the arguments do? What modifications do you need to make for the \texttt{section5} package?}

%into the package you made in section 2
% \begin{minted}{text}
% ~/catkin_ws/src/aa274_s2/rviz/my_nav.rviz
% \end{minted}

% Before starting, have someone else working with the robot ready to run the teleop node to take over control and stop the robot if necessary
% \begin{minted}{bash}
% roslaunch turtlebot3_teleop turtlebot3_teleop_key.launch
% \end{minted}
% (Don't actually run this until you need to! This overrides any commands to the turtlebot so your navigator won't work if teleop is running.)

Now you can specify goal poses using the ``2D Nav Goal'' button in rviz and clicking and dragging on the map. The robot should move towards the goal if your controllers work correctly!

% \textbf{Problem 3: Test this out on the robot (or simulation). Include a screenshot of rviz as your robot navigates the map.}
\textbf{Problem 4: Test this out. Include a screenshot of rviz as your robot navigates the map. }

\section{Visualizing the goal position}
Using what you learned in last section, write a new node that visualizes the current navigation target in rviz as a marker. Save this node in the \texttt{section5} package's \texttt{scripts} folder.

\textbf{Problem 5: Describe at a high level how your goal visualizer works. Some questions to get you started are: 
\begin{itemize}
    \item What topics should it subscribe to in order to stay up to date with the current navigation target?
    \item What message type should it publish, and to what topic?
\end{itemize} 
Include this code in your submission.}

\section{Custom Launch Files}
It can be cumbersome to start all the nodes from scratch, and set up rviz every time we want to run the stack. To make this easier, create a launch file in your \texttt{section5} package which:
\begin{enumerate}
    \item starts the \texttt{navigator.py} node from the \texttt{asl\_turtlebot} package.
    \item starts the goal visualization node you just wrote.
    \item opens \texttt{rviz} with the configuration file you just saved.
\end{enumerate}

Hint: run \texttt{rviz --help} to see how to pass a configuration file into rviz.
Use the ROS documentation and/or Google to find out how to pass arguments into nodes through a launch file.

Take a look at the launch files in the \texttt{asl\_turtlebot} package. In particular, lines \texttt{5 - 8} of the \texttt{asl\_turtlebot asl\_turtlebot\_core.launch} file also provide an example of starting each of these nodes and it also provides a minimal example of a launch file.

Once you've written your launch file, save it as


\begin{lstlisting}
~/catkin_ws/src/section5/launch/my_nav.launch
\end{lstlisting}


Test it out by running


\begin{lstlisting}
roslaunch section5 my_nav.launch
\end{lstlisting}


\textbf{Problem 6: Describe the components included in your launch file. Did you use any of the \texttt{asl\_turtlebot} launch files as an example? If so, what changes did you make? Include the contents of this launch file in your submission}

\end{document}
